\chapter{Guia 1 Ejercicio 12}

Considerando el circuito de la figura 12 antes de que la llave cambie de posición
(o sea, en $t = 0^{-}$ ), calcular el valor de la corriente del inductor $i_L(0^-) = I_{L0}$ y de
tensión en el capacitor $v_C (0^-) = VC0$ . Calcular la solución numérica con $V = 10\V$ ,
$R1 = 100\ohm$, $R2 = 270\ohm$, $L = 200\mH$, y $C = 0,1\F$.

\begin{figure}[h]
  \centering
  \begin{circuitikz}
    \ctikzset{resistors/scale=0.7}
    \ctikzset{capacitors/scale=0.7}
    \ctikzset{v/.append style={/tikz/american voltages}}
    \draw (0,0) to[battery1, l_=$V$] ++(0,-2.5)
      to[L, l_=$L$, f<_=$i_L$] ++(2.5,0) coordinate(L)
      to[R, l_=$R_2$] ++(0,2.5) coordinate(R1)
      to[R, l_=$R_1$] ++(-2.5,0);
    \draw (R1) ++(2,-0.265) node[cute spdt up arrow, rotate=90, scale=0.75](SW){} ;
    \draw (SW) ++(0,.5) node[](){$t=0$} ;
    \draw (R1) to[short] (SW.out 1);

    \draw (L) to[short] (SW.in |- L)
      to[short] ++(0,0.4)
      to[C=$C$, v<=$v_C$] (SW.in);

    \draw (L) to[short] (SW.in |- L)
      to[short] ++(2,0) coordinate (R3)
      to[R, l_=$R$] (R3 |- SW.out 2)
      to[short] (SW.out 2);

  \end{circuitikz}
  \caption{Circuito}
\end{figure}

Para $t < 0$ el circuito puede simplificarse a:

\begin{figure}[h]
  \centering
  \begin{circuitikz}
    \ctikzset{resistors/scale=0.7}
    \ctikzset{capacitors/scale=0.7}
    \ctikzset{v/.append style={/tikz/american voltages}}
    \draw (0,0) to[battery1, l_=$V$] ++(0,-2.5)
      to[L, l_=$L$, f<_=$\tiny i_L$, v^<=$v_{L}$] ++(2.5,0) coordinate(L)
      to[R, l_=$R_2$, f_<=$\tiny i_{R_2}$] ++(0,2.5) coordinate(R1)
      to[R, l_=$R_1$, v^<=$v_{R_1}$] ++(-2.5,0);
    \draw (L) to[short] ++(2.5,0)
      to[C=$C$, v<=$v_C$] ++(0,2.5)
      to[short, f_<=$\tiny i_C$] (R1);
  \end{circuitikz}
  \caption{Circuito para $t<0$}
\end{figure}

El sistema de ecuaciones que modela el comportamiento del circuito es:

\begin{equation}
  \begin{cases}
    V - v_{R_1} - v_C - v_L = 0\\
    v_{R_1} = i_L R_1\\
    i_C = C \dfrac{dv_C}{dt} \\
    v_c = i_{R_2} R_2\\
    v_L = L\dfrac{di_L}{dt}
  \end{cases}
\end{equation}

Para el instante $t=0^-$ podemos considerar el régimen transitorio extinto. Y como la única fuente de excitación del circuito es constante
podemos considerar que las tensiones y corrientes del circuito también lo son,y por ende:

\begin{gather}
  \frac{dv_C}{dt} \Big|_{t=0^-} = 0 \quad \Rightarrow \quad i_C(0^-) = 0\\
  \frac{di_L}{dt} \Big|_{t=0^-} = 0 \quad \Rightarrow \quad v_L(0^-) = 0
\end{gather}

Con estas consideraciones, podemos reescribir el circuito como:


\begin{figure}[h]
  \centering
  \begin{circuitikz}
    \ctikzset{resistors/scale=0.7}
    \ctikzset{capacitors/scale=0.7}
    \ctikzset{v/.append style={/tikz/american voltages}}
    \draw (0,0) to[battery1, l_=$V$] ++(0,-2.5)
      to[short, f<_=$\tiny i_L$,  v^<=$v_{L}$] ++(2.5,0) coordinate(L)
      to[R, l_=$R_2$, f_<=$\tiny i_{R_2}$] ++(0,2.5) coordinate(R1)
      to[R, l_=$R_1$, v^<=$v_{R_1}$] ++(-2.5,0);
    \draw (L) to[short] ++(2.5,0)
      to[open, o-o, v<=$v_C$] ++(0,2.5)
      to[short, f_<=$\tiny i_C$] (R1);
  \end{circuitikz}
  \caption{Circuito para $t=0^-$}
\end{figure}

Podemos ahora calcular $i_L(0^-)$  y $v_c(0^-)$ como:

\begin{equation}
  i_L(0^-) = \frac{V}{R_1 + R_2}
\end{equation}

\begin{equation}
  v_C(0^-) = i_L (0^-) R_2 =\frac{V}{R_1 + R_2}R_2
\end{equation}

Finalmente, para $V = 10\V$ , $R1 = 100\ohm$, $R2 = 270\ohm$,

\begin{equation}
  i_L(0^-) = \frac{10\V}{100\ohm + 270\ohm} = 27.027 \mA
\end{equation}

\begin{equation}
  v_C(0^-) =\frac{10\V}{100\ohm + 270\ohm}270\ohm = 7.297 \V
\end{equation}


