\chapter{Guia 2 Ejercicio 1}

Calcular el valor medio, valor eficaz y factor de forma de las señal de excitación:


\pgfmathsetmacro{\I}{1}
\pgfmathsetmacro{\T}{2}
\begin{figure}[h]
  \centering
  \begin{tikzpicture}
    \begin{axis}[
        axis lines=middle,
        xlabel={$t$},
        ylabel={[I]},
        xtick={1,2},
        xticklabels={\small$\frac{T}{2}$,\small$T$},
        ytick={-2,-1,1,2},
        yticklabels={$-2I_m$, , $I_m$},
        domain=0:3,
        samples=200,
        grid=both,
        width=7cm,
        height=6cm,
        ymin=-2.5, ymax=2.5,
        xmin=0, xmax=3,
      ]
      \addplot[blue, thick,samples at={0,1}, restrict x to domain=0:1] {(6*\I/\T)* x -2*\I};
      \addplot[blue, thick,samples at={1,2}, restrict x to domain=1:2] {-(6*\I/\T)* x +4*\I};
      \addplot[blue, thick,samples at={2,3}, restrict x to domain=2:3] {(6*\I/\T)* x -8*\I};
    \end{axis}
  \end{tikzpicture}
  \caption{Señal}
\end{figure}

Podemos definir por partes un periodo de la señal como:

\begin{equation}
i(t) =
  \begin{cases}
    \frac{3I_m}{\frac{T}{2}} t - 2I_m \quad 0>t>\frac{T}{2}\\\\
    -\frac{3I_m}{\frac{T}{2}} t + 4I_m \quad \frac{T}{2}>t>T\\
  \end{cases}
\end{equation}

\section{valor medio}

El valor medio de la señal esta definido por:

\begin{equation}
  I_{med}= \frac{1}{T} \int_0^T i(t) dt
\end{equation}

Para calcular la integral, podemos sumar el área de los dos triángulos formados por la curva.

\begin{figure}[H]
  \centering
  \begin{tikzpicture}
    \begin{axis}[
        axis lines=middle,
        xlabel={$t$},
        ylabel={[I]},
        xtick={1/3,2/3,1,4/3,5/3,2},
        xticklabels={,\small$\frac{T}{3}$,\small$\frac{T}{2}$,,,\small$T$},
        ytick={-2,-1,1,2},
        yticklabels={$-2I_m$, , $I_m$},
        domain=0:3,
        samples=200,
        grid=both,
        width=7cm,
        height=6cm,
        ymin=-2.5, ymax=2.5,
        xmin=0, xmax=2,
      ]
      \addplot[blue, thick,samples at={0,1}, restrict x to domain=0:1, name path=f] {(6*\I/\T)* x -2*\I};
      \addplot[blue, thick,samples at={1,2}, restrict x to domain=1:2] {-(6*\I/\T)* x +4*\I};
      \path[name path=axis] (axis cs:0,0) -- (axis cs:1,0);
      \addplot[ fill=blue!20, fill opacity=0.5, ] fill between[of=f and axis];
      \node at (axis cs:0.2,-0.8) {$\small A_1$};
      \node at (axis cs:0.87,0.22) {$\small A_2$};
    \end{axis}
  \end{tikzpicture}
  \caption{Areas bajo la curva}
\end{figure}

\begin{equation}
A_1=-2I_m\frac{T}{3}\frac{1}{2} \qquad A_2=I_m\frac{T}{6}\frac{1}{2}
\end{equation}

Con esto, podemos calcular la integral como:

\begin{equation}
  \begin{aligned}
    \int_0^T i(t)dt &= 2(A_1+A_2)\\
    \int_0^T i(t)dt &= -2I_m\frac{T}{3} + I_m\frac{T}{6}\\
    \int_0^T i(t)dt &= -\frac{I_mT}{2}\\
  \end{aligned}
\end{equation}

Finalmente:

\begin{equation}
  I_{med}= \frac{1}{T} \int_0^T i(t) dt = -\frac{I_m}{2}
\end{equation}

\section{valor eficaz}

El valor eficaz de la señal esta definido por:

\begin{equation}
  I_{ef}= \sqrt{ \frac{1}{T} \int_0^T i^2(t) dt }
\end{equation}

Para calcular la integral, podemos nuevamente aprovecharnos de la simetría de la señal, planteando:

\begin{equation}
  \int_0^T i^2(t) dt = 2 \int_0^{\frac{T}{2}} i^2(t) dt
\end{equation}

Calculamos la nueva integral:

\begin{align*}
  \int_0^{\frac{T}{2}} i^2(t) dt &= \int_0^{\frac{T}{2}} \left(\frac{3I_m}{T/2} t - 2I_m\right)^2 dt\\
  &= \int_0^{\frac{T}{2}} \frac{9I_m^2}{\frac{T^2}{4}} t^2 - \frac{12 I_m^2}{\frac{T}{2}} t + 4I_m^2 dt\\
  &= \frac{9I_m^2}{\frac{T^2}{4}} \int_0^{\frac{T}{2}} t^2 dt - \frac{12 I_m^2}{\frac{T}{2}} \int_0^{\frac{T}{2}} t dt + 4I_m^2 \int_0^{\frac{T}{2}} dt\\
  &= \frac{9I_m^2}{\frac{T^2}{4}} \frac{t^3}{3} \bigg|_0^{\frac{T}{2}} - \frac{12 I_m^2}{\frac{T}{2}} \frac{t^2}{2} \bigg|_0^{\frac{T}{2}} + 4I_m^2 t \Big|_0^{\frac{T}{2}} \\
  &= \frac{9I_m^2}{\frac{T^2}{4}} \frac{\left(\frac{T}{2}\right)^3}{3} - \frac{12 I_m^2}{\frac{T}{2}} \frac{\left(\frac{T}{2}\right)^2}{2} + 4I_m^2 \frac{T}{2} \\
  &= \frac{3I_m^2}{\frac{T^2}{4}} \frac{T^3}{8} - \frac{6 I_m^2}{\frac{T}{2}} \frac{T^2}{4} + 2I_m^2 T \\
  &= \frac{3}{2} I_m^2T - \frac{6}{2} I_m^2 T + 2I_m^2 T \\
  \int_0^{\frac{T}{2}} i^2(t) dt &= \frac{1}{2} I_m^2 T\\
\end{align*}

Finalmente:

\begin{equation}
  I_{ef}= \sqrt{ \frac{1}{T} \int_0^T i^2(t) dt } = \sqrt{ \frac{1}{T} 2 \frac{1}{2} I_m^2 T} = I_m
\end{equation}

%Separando la integral en los dos intervalos en los que se define la señal para un periodo y operando:
%
%\begin{equation}
%  \begin{aligned}
%    I_{med} &= \frac{1}{T} \left(\int_0^{\frac{T}{2}} \frac{3I_m}{T/2} t - 2I_m dt + \int_\frac{T}{2}^{T} -\frac{3I_m}{T/2} t + 4I_m dt\right)\\
%    I_{med} &= \frac{1}{T} \left(\int_0^{\frac{T}{2}} \frac{3I_m}{T/2} t dt - \int_0^{\frac{T}{2}} 2I_m dt + \int_\frac{T}{2}^{T} -\frac{3I_m}{T/2} t dt
%      + \int_\frac{T}{2}^{T}4I_m dt\right)\\
%    I_{med} &= \frac{1}{T} \left(\frac{3I_m}{T/2} \int_0^{\frac{T}{2}} t dt - 2I_m \int_0^{\frac{T}{2}} dt -\frac{3I_m}{T/2} \int_\frac{T}{2}^{T} t dt
%      + 4I_m \int_\frac{T}{2}^{T} dt\right)\\
%    I_{med} &= \frac{1}{T} \left(\frac{3I_m}{T/2} \frac{t^2}{2} \Big|_0^{\frac{T}{2}} - 2I_m t\Big|_0^{\frac{T}{2}} -\frac{3I_m}{T/2} \frac{t^2}{2}\Big|_\frac{T}{2}^{T}
%      + 4I_m t\Big|_\frac{T}{2}^{T} \right)\\
%    I_{med} &= \frac{1}{T} \left(\frac{3I_m}{T/2} \frac{\left(\frac{T}{2}\right)^2}{2}- 2I_m \frac{T}{2} -\frac{3I_m}{T/2} \frac{t^2}{2}\Big|_\frac{T}{2}^{T}
%      + 4I_m t\Big|_\frac{T}{2}^{T} \right)\\
%  \end{aligned}
%\end{equation}













